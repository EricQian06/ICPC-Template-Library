\documentclass[12pt, a4paper, oneside]{ctexart}
\usepackage{minted, multicol, geometry, graphicx, fancyvrb}
\usepackage{relsize, setspace, enumitem, float, hyperref}

\geometry{a4paper, scale = 0.85}

\setenumerate[1]{itemsep=0pt,partopsep=0pt,parsep=\parskip,topsep=0pt}
\setitemize[1]{itemsep=0pt,partopsep=0pt,parsep=\parskip,topsep=0pt}
\setdescription{itemsep=0pt,pa
rtopsep=0pt,parsep=\parskip,topsep=0pt}
\setlength{\parindent}{0pt}

\setminted[cpp]{
	style=xcode,
	mathescape,
	linenos,
	autogobble,
	baselinestretch=1,
	tabsize=3,
	fontsize=\scriptsize,
	%bgcolor=Gray,
	frame=single,
	framesep=1mm,
	framerule=0.3pt,
	numbersep=1mm,
	breaklines=true,
	breaksymbolsepleft=2pt,
	%breaksymbolleft=\raisebox{0.8ex}{ \small\reflectbox{\carriagereturn}}, %not moe!
	%breaksymbolright=\small\carriagereturn,
	breakbytoken=false,
	% showtabs=true,
	% tab={\relscale{0.6} $\big\vert \ \ \ $ \relscale{1}},
}

\title{ACM 模板}
\author{钱智煊,黄佳瑞,车昕宇}
\date{\today}

\begin{document}
    \scriptsize
    \maketitle
    \newpage
    
    \begin{multicols}{2}
        \tableofcontents
        \newpage

        \section{图论}
        \subsection{tarjan}
        \subsubsection{有向图缩点}
        \inputminted{cpp}{src/graph/tarjan-directed.cpp}
        \subsubsection{无向图求割点}
        \inputminted{cpp}{src/graph/tarjan-vertex.cpp}
        \subsubsection{无向图求桥}
        \inputminted{cpp}{src/graph/tarjan-edge.cpp}
        \subsubsection{构建圆方树}
        将点双连通分量定义为不含割点的联通图。

将原图中的点称作圆点,而每个点双对应一个方点。将每个圆点向其所在的所有点双对应方点连边,就得到了圆方树。

特别的,仅含一条边的图也是点双(因其不含割点),这保证了圆方树的连通性。

\begin{figure}[htbp]
    \centering
    \includegraphics[width=0.8\textwidth]{https://pic.imgdb.cn/item/66d46de3d9c307b7e9acf664.png}
    \caption{圆方树示意图}
\end{figure}

圆方树会建立很多新的点,所以不要忘记\textbf{给数组开两倍!}

\begin{minted}{cpp}
void tarjan(int u)
{
    dfn[u]=low[u]=++Time,sta[++tp]=u;
    for(int v:G[u])
    {
        if(!dfn[v])
        {
            tarjan(v),low[u]=min(low[u],low[v]);
            if(low[v]==dfn[u])
            {
                int hav=0; ++All;
                for(int x=0;x!=v;tp--) x=sta[tp],T[x].pb(All),T[All].pb(x),hav++;
                T[u].pb(All),T[All].pb(u),;
                siz[All]=++hav;
            }
        }
        else low[u]=min(low[u],dfn[v]);
    }
}
\end{minted}
        \subsubsection{2-SAT}
        2-sat 问题定义为:给定 $m$ 个布尔表达式,每个包含两个布尔变量,形如 $a\lor b$ 。求是否存在一种赋值方式,使得所有表达式都为真。

对于表达式 $a\lor b$ ,显然若 $a$ 假则 $b$ 必真,反之亦然。因此,考虑将每个变量拆为两个点,分别代表此变量取值为真或假。对于每个表达式 $a\lor b$ ,连边 $\neg a\to b$ 与 $\neg b\to a$ 。对于边 $u\to v$ ,意义为若 $u$ 为真则 $v$ 必为真。那么对于最后得到的图,如果同一变量拆出的点处在同一强连通分量中,则无解,因为在可行解中它们的取值必然是不同的,但同一强连通分量中的点取值必然相同;否则,考虑缩点以后得到的 DAG ,取其拓扑序,对于每一个变量,令其取拆出的拓扑序较大的点对应的值即可。特别的,以上的判定是必要的,但我还不知道怎么证明它是充分的。
        \subsection{欧拉路径}
        欧拉图的判定:考虑起点、终点与中间点的必要条件即可。可以证明必要条件也是充分的。
以下给出已确定起点的无向图欧拉路径(回路)的构造算法。
\inputminted{cpp}{src/graph/euler-path.cpp}
        % \subsection{斯坦纳树}
        \subsection{网络流}
        \subsubsection{最大流(dinic)}
        \inputminted{cpp}{src/graph/dinic.cpp}
        \subsubsection{最小费用最大流(dinic)}
        \inputminted{cpp}{src/graph/mcmf-dinic.cpp}
        \subsubsection{最小费用最大流(edmonds-karp)}
        \inputminted{cpp}{src/graph/mcmf-ek.cpp}
        % \subsubsection{最小割树}

        \section{数据结构}
        \subsection{平衡树}
        \subsubsection{无旋 Treap}
        \inputminted{cpp}{src/data structure/fhq.cpp}
        \subsubsection{平衡树合并}
        如果需要合并两个有交集的 Treap 时该怎么做?我们可以每次将较小的数合并到较大的树中去,这样每个点最多只会合并 $\log n$ 次,每次合并复杂度 $O(n\log n)$,总时间复杂度 $O(n\log n\log V)$。

代码其实非常暴力,就是直接对更小的那棵树直接一个个插入进去:

\inputminted{cpp}{src/data structure/treap-merge.cpp}

\href{https://codeforces.com/blog/entry/108601}{可以证明},若只支持合并与分裂操作,则时间复杂度为 $O(n\log n)$ 。
        \subsubsection{Splay}
        \inputminted{cpp}{src/data structure/splay.cpp}
        \subsection{动态树}
        \inputminted{cpp}{src/data structure/LCT.cpp}
        \subsection{珂朵莉树}
        \inputminted{cpp}{src/data structure/ODT.cpp}
        \subsection{李超线段树}
        \subsubsection{修改与询问}
        \inputminted{cpp}{src/data structure/lichao.cpp}
        \subsubsection{合并}
        \inputminted{cpp}{src/data structure/lichao-merge.cpp}
        \subsection{二维树状数组}
        \inputminted{cpp}{src/data structure/2D-BIT.cpp}
        \subsection{虚树}
        \inputminted{cpp}{src/data structure/virtual-tree.cpp}
        \subsection{左偏树}
        \inputminted{cpp}{src/data structure/heap.cpp}
特别的,若要求给定节点所在左偏树的根,须使用并查集。对于每个节点维护 rt[] 值,查找根时使用函数:
\begin{minted}{cpp}
int find(int x) { return rt[x] == x ? x : rt[x] = find(rt[x]); }
\end{minted}
在合并节点时,加入:
\begin{minted}{cpp}
rt[x] = rt[y] = merge(x, y);
\end{minted}
在弹出最小值时加入:
\begin{minted}{cpp}
rt[ls(x)] = rt[rs(x)] = rt[x] = merge(ls(x), rs(x));
\end{minted}
另外,删除过的点是不能复用的,因为这些点可能作为并查集的中转节点。
        \subsection{吉司机线段树}
        \begin{itemize}
    \item 区间取 min 操作:通过维护区间次小值实现,即将区间取 min 转化为对区间最大值的加法,当要取 min 的值 v 大于次小值时停止递归。时间复杂度通过标记回收证明,即将区间最值视作标记,这样每次多余的递归等价于标记回收,总时间复杂度为 $O(m\log n)$。
    \item 区间历史最大值:通过维护加法标记的历史最大值实现。
\end{itemize}
\inputminted{cpp}{src/data structure/seg-beats.cpp}
        \subsection{树分治}
        熟知序列分治的过程是选取恰当的分治点并考虑所有跨过分治点的区间。而树分治的过程也是类似的,以点分治为例,每一次选择当前联通块的重心作为分治点,然后考虑所有跨越分治点的路径,并对分割出的联通块递归。

若要处理树上邻域问题,可以考虑建出点分树。处理点 x 的询问时,只需考虑 x 在点分树上到根的路径,每一次加上除开 x 所在子树的答案即可。

\inputminted{cpp}{src/data structure/tree-divide.cpp}
        % 关于扫描线。
        % 线段树优化建图没有必要。
        % 线段树分治没有必要。
        % 莫队的奇偶优化(看情况)
        % 笛卡尔树计数(为什么要写这种东西)
        % 树分块。
        % 树状数组上二分。
        % ETT 。
        % cdq 以及其它分治。
        % 莫队。

        % exchange argument 。
        
        \section{字符串}
        \subsection{后缀数组(与后缀树)}
        \inputminted{cpp}{src/string/SA.cpp}
        \subsection{AC自动机}
        \inputminted{cpp}{src/string/ACAM.cpp}
        \subsection{回文自动机}
        \inputminted{cpp}{src/string/PAM.cpp}
        \subsection{Manacher算法}
        \inputminted{cpp}{src/string/manacher.cpp}
        \subsection{KMP算法与border理论}
        \inputminted{cpp}{src/string/kmp.cpp}
字符串的border理论:
以下记字符串 $S$ 的长度为 $n$ 。
\begin{itemize}
    \item 若串 $S$ 具备长度为 $m$ 的 border ,则其必然具备长度为 $n-m$ 的周期,反之亦然。
    \item 弱周期性引理:若串 $S$ 存在周期 $p$ 、$q$ ,且 $p+q\le n$ ,则 $S$ 必然存在周期 $\gcd(p,q)$ 。
    \item 引理1:若串 $S$ 存在长度为 $m$ 的 border $T$,且 $T$ 具备周期 $p$ ,满足 $2m-n\ge p$ ,则 $S$ 同样具备周期 $p$ 。
    \item 周期性引理:若串 $S$ 存在周期 $p$ 、$q$ ,满足 $p+q-\gcd(p,q)\le n$ ,则串 $S$ 必然存在周期 $\gcd(p,q)$ 。
    \item 引理2:串 $S$ 的所有 border 的长度构成了 $O(\log n)$ 个不交的等差数列。更具体的,记串 $S$ 的最小周期为 $p$ ,则其所有长度包含于区间 $[n \bmod p + p, n)$ 的 border 构成了一个等差数列。
    \item 引理3:若存在串 $S$ 、$T$ ,使得 $2|T|\ge n$ ,则 $T$ 在 $S$ 中的所有匹配位置构成了一个等差数列。
    \item 引理4:PAM 的失配链可以被划分为 $O(\log n)$ 个等差数列。
\end{itemize}
        \subsection{Z函数}
        Z函数用于求解字符串的每一个后缀与其本身的 lcp 。其思路和 manacher 算法基本一致,都是维护一个扩展过的最右端点和对应的起点,而当前点要么暴力扩展使最右端点右移,要么处在记录的起点和终点间,从而可以利用已有的信息快速转移。
\inputminted{cpp}{src/string/zfunc.cpp}

        \section{数学}
        % 记得写格林公式之类的。
        % 还有单纯形法。

        \section{多项式}
        \subsection{FFT}
        \inputminted{cpp}{src/poly/fft.cpp}
        \subsection{NTT}
        \inputminted{cpp}{src/poly/ntt.cpp}
        \subsection{集合幂级数}
        \subsubsection{并卷积、交卷积与子集卷积}
        集合并等价于二进制按位或,因此并卷积的计算实际上就是做高维前缀和以及差分,也被称作莫比乌斯变换。
\inputminted{cpp}{src/poly/or.cpp}
而集合交卷积则对应后缀和。
\inputminted{cpp}{src/poly/and.cpp}
子集卷积则较为特殊,为了使得产生贡献的集合没有交集,考虑引入代表集合大小的占位符。这样只需做 $n$ 次 FMT ,再枚举长度做 $n^2$ 次卷积。因为 FMT 具备线性性,所以最后只需做 $n$ 次 iFMT 即可。
\inputminted{cpp}{src/poly/linear.cpp}
特别的,子集卷积等价于 $n$ 元保留到一次项的线性卷积。
        \subsubsection{对称差卷积}
        集合对称差等价于按位异或,而异或卷积则等价于 $n$ 元模 $2$ 的循环卷积,因此,FWT 实质上和 $n$ 元 FFT 没有什么区别。
\inputminted{cpp}{src/poly/xor.cpp}
        \subsection{多项式全家桶}
        \inputminted{cpp}{src/poly/poly.cpp}

        \section{杂项与结论}
        \subsection{对拍器}
        \subsubsection{linux}
        \inputminted{cpp}{src/tools/linux_checker.cpp}
        \subsubsection{windows}
        \inputminted{cpp}{src/tools/win_checker.cpp}
        \subsection{阴间错误集锦}
        % 构建表达式树
    \end{multicols}
\end{document}