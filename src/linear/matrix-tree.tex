以下叙述允许重边,不允许自环。

对于无向图 $G$ ,定义度数矩阵 $D$ 为:
$$
    D_{ij} = \deg(i)[i=j]
$$
设 $\#e(i,j)$ 为连接点 $i$ 和 $j$ 的边数,定义邻接矩阵 $A$ 为:
$$
    A_{ij} = \#e(i,j)
$$
显然 $A_{ii}=0$ 。
定义 Laplace 矩阵 $L$ 为 $D-A$ ,记 $G$ 的生成树个数为 $t(G)$ ,则其恰为 $L$ 的任意一个 $n-1$ 阶主子式的值。

对于有向图 $G$ ,分别定义出度矩阵 $D^{out}$ 和入度矩阵 $D^{in}$ 为:
$$
    \begin{aligned}
        D^{out}_{ij} & = \deg^{out}(i)[i=j] \\
        D^{in}_{ij}  & = \deg^{in}(i)[i=j]
    \end{aligned}
$$
设 $\#e(i,j)$ 为从点 $i$ 到 $j$ 的边数,定义邻接矩阵 $A$ 为:
$$
    A_{ij} = \#e(i,j)
$$
显然 $A_{ii}=0$ 。
再分别定义出度 Laplace 矩阵 $L^{out}$ 和入度 Laplace 矩阵 $L^{in}$ 为:
$$
    \begin{aligned}
        L^{out} & = D^{out}-A \\
        L^{in}  & = D^{in}-A
    \end{aligned}
$$
分别记 $G$ 的以 $k$ 为根的根向树形图个数为 $t^{root}(k)$ ,以及以 $k$ 为根的叶向树形图个数为 $t^{leaf}(k)$ 。则 $t^{root}(k)$ 恰为 $L^{out}$ 的删去 $k$ 行 $k$ 列的 $n-1$ 阶主子式的值;$t^{leaf}(k)$ 恰为 $L^{in}$ 的删去 $k$ 行 $k$ 列的 $n-1$ 阶主子式的值。