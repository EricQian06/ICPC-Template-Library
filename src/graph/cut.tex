注意这里的 $dfn$ 表示\textbf{不经过父亲},能到达的最小的 $dfn$ 。
割点:
\begin{itemize}
    \item 若 $u$ 是根节点,当至少存在 $2$ 条边满足 $low(v)\ge dfn(u)$ 则 $u$ 是割点 。
    \item 若 $u$ 不是根节点,当至少存在 $1$ 条边满足 $low(v)\ge dfn(u)$ 则 $u$ 是割点 。
\end{itemize}
割边:
\begin{itemize}
    \item 当存在一条边条边满足 $low(v)>dfn(u)$ 则边 $i$ 是割边。
\end{itemize}
注意:
记录上一个访问的边时要记录边的编号,不能记录上一个过来的节点(因为会有重边)!!!或者在加边的时候特判一下,不过注意编号问题。(用输入顺序来对应数组中位置的时候,重边跳过,但是需要 tot+=2)