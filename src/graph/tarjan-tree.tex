将点双连通分量定义为不含割点的联通图。

将原图中的点称作圆点,而每个点双对应一个方点。将每个圆点向其所在的所有点双对应方点连边,就得到了圆方树。

特别的,仅含一条边的图也是点双(因其不含割点),这保证了圆方树的连通性。

\begin{figure}[htbp]
    \centering
    \includegraphics[width=0.8\textwidth]{https://pic.imgdb.cn/item/66d46de3d9c307b7e9acf664.png}
    \caption{圆方树示意图}
\end{figure}

圆方树会建立很多新的点,所以不要忘记\textbf{给数组开两倍!}

\begin{minted}{cpp}
void tarjan(int u)
{
    dfn[u]=low[u]=++Time,sta[++tp]=u;
    for(int v:G[u])
    {
        if(!dfn[v])
        {
            tarjan(v),low[u]=min(low[u],low[v]);
            if(low[v]==dfn[u])
            {
                int hav=0; ++All;
                for(int x=0;x!=v;tp--) x=sta[tp],T[x].pb(All),T[All].pb(x),hav++;
                T[u].pb(All),T[All].pb(u),;
                siz[All]=++hav;
            }
        }
        else low[u]=min(low[u],dfn[v]);
    }
}
\end{minted}