\paragraph*{itertools 库}

\begin{minted}[]{python3}
from itertools import *
# 笛卡尔积
product('ABCD', 'xy') # Ax Ay Bx By Cx Cy Dx Dy
product(range(2), repeat=3) # 000 001 010 011 100 101 110 111
# 排列
permutations('ABCD', 2) # AB AC AD BA BC BD CA CB CD DA DB DC
# 组合
combinations('ABCD', 2) # AB AC AD BC BD CD
# 有重复的组合
combinations_with_replacement('ABC', 2) # AA AB AC BB BC CC
\end{minted}

\paragraph*{random 库}
\begin{minted}[]{python3}
from random import * 
randint(l, r) # 在 [l, r] 内的随机整数
choice([1, 2, 3, 5, 8]) # 随机选择序列中一个元素
sample([1, 2, 3, 4, 5], k=2) # 随机抽样两个元素
shuffle(x) # 原地打乱序列 x
l,r = sorted(choices(range(1, N+1), k=2)) # 生成随机区间 [l,r]
binomialvariate(n, p) # 返回服从 B(n,p) 的一个变量
normalvariate(mu, sigma) # 返回服从 N(mu,sigma) 的一个变量
\end{minted}

\paragraph*{列表操作}
\begin{minted}[]{python3}
# 列表操作
l = sample(range(100000), 10)
l.sort() # 原地排序
l.sort(key=lambda x:x%10) # 按末尾排序
from functools import cmp_to_key
l.sort(key=cmp_to_key(lambda x,y:y-x)) # 比较函数,小于返回负数
sorted(l) # 非原地排序
l.reverse() reversed(l)
\end{minted}
\paragraph*{字典操作}
\begin{minted}[]{python3}
from collections import defaultdict
# 提供一个函数返回缺省值
d = defaultdict(list)
d["a"].append(2)
d["a"].append(3)
d["b"].append(4)
print(d) # {'a': [2, 3], 'b': [4]}
# 用 lambda 可以快速构造出需要的默认值
d = defaultdict(lambda: 2)
# 遍历键值对
for k,v in d.items():
    print(k, v)
\end{minted}

\paragraph*{复数}
\begin{minted}[]{python3}
a = 1+2j
print(a.real, a.imag, abs(a), a.conjugate())
\end{minted}

\paragraph*{高精度小数}
\begin{minted}[]{python3}
from decimal import Decimal, getcontext, FloatOperation, ROUND_HALF_EVEN
getcontext().prec = 100 # 设置有效位数
getcontext().rounding = getattr(ROUND_HALF_EVEN) # 四舍六入五成双
getcontext().traps[FloatOperation] = True # 禁止 float 混合运算
a = Decimal("114514.1919810")
print(a, f"{a:.2f}")
a.ln() a.log10() a.sqrt() a**2
\end{minted}


\paragraph*{记忆化搜索}
\begin{minted}[]{python3}
from functools import cache
# 记忆化搜索,还可以记忆化元组,只要参数满足 Hashable 即可
@cache 
def fib(n):
    if n<=2:
        return 1
    return fib(n-1)+fib(n-2)
\end{minted}